\section*{ВВЕДЕНИЕ}
\addcontentsline{toc}{section}{ВВЕДЕНИЕ}

За последние несколько десятилетий, с появлением Youtube, Amazon,\\Netflix и многих других подобных веб–сервисов, системы рекомендаций стали занимать все больше места в нашей жизни. Начиная с электронной коммерции (предлагая статьи, которые могут заинтересовать людей) и заканчивая рекламой в Интернете.

Многие современные сервисы создают рекомендательные системы, которые основываясь на информации о пользователе и его поведении в системе, пытаются определить какие объекты ему интересны, будь то товары, новости, услуги  и т.д. Яркими примерами служат такие сервисы или сайты, как «КиноПоиск», «Яндекс.Дзен», «Яндекс.Новости» и многие другие. «КиноПоиск» --- российский веб-сайт, предлагающий пользователю к просмотру фильмы на основе его предпочтений. «Яндекс.Дзен» --- веб-сайт и расширение для браузера от компании «Яндекс», ищущее в интернете информацию, которая может быть интересна пользователю, и собирающее ее в персональную ленту. «Яндекс.Новости» --- российский веб-сайт, предлагающий к просмотру новости от партнеров службы, в числе которых ведущие российские и зарубежные СМИ. Поступающая информация автоматически группируется в сюжеты. На их основе формируется информационная картинка дня.

Как видно из примеров, рекомендательные системы, улучшают пользовательский опыт, упрощают нахождение наиболее интересного для пользователя контента. Поэтому со временем, количество сервисов применяющих рекомендательные системы растет и начинает широко применятся во многих сферах, таких как электронная коммерция, при поиске фильмов, музыки, научных статей, а также на новостных сайтах и в справочных центрах, а задача разработки эффективных рекомендательных систем является актуальной.

Выделяют два основных метода построения рекомендательных систем --- метод фильтрации на основе содержания и метод коллаборативной фильтрации.

Методы фильтрации на основе содержания основаны на описании\\объекта и профиле предпочтений пользователя. Данный подход пытается подобрать объекты, похожие на те, что нравились пользователю ранее, и опирается на методы информационного поиска и машинного обучения.

Метод коллаборативной фильтрации базируется на информации об истории поведения всех пользователей в системе. К примеру, если это сайт по продаже электроники, то рекомендация по покупке товаров основывается на пользователях со схожей историей и их отношениях к объекту.

Одним из направлений обработки данных различной структуры и свойств является кластеризация. Существует множество методов кластеризации, которые можно классифицировать как четкие и нечеткие. Четкие методы кластеризации разбивают исходное множество объектов на несколько непересекающихся подмножеств. При этом любой объект принадлежит только одному кластеру. Нечеткие методы кластеризации позволяют одному и тому же объекту принадлежать одновременно нескольким (или даже всем) кластерам, но с различной степенью принадлежности. Нечеткая кластеризация во многих ситуациях более “естественна”, чем четкая, например, для объектов расположенных на границе кластеров.

В данной работе должны быть проанализированы методы нечеткой кластеризации, существующие подходы к рекомендательным системам, а также потребуется разработать рекомендательную систему с применением алгоритма нечеткой кластеризации.

Целью работы является разработка и реализация рекомендательной системы новостей на основе нечеткой кластеризации методом Гауссовой смеси, а также проведение исследования работоспособности реализованной рекомендательной системы и разработанного алгоритма нечеткой кластеризации.

Для достижения поставленной цели необходимо решить следующие задачи:
\begin{itemize}
	\item провести анализ предметной области, выделить основные определения;
	\item провести анализ существующих подходов реализации рекомендательных
	систем;
	\item выделить основные критерии для сравнения и выбора наиболее
	подходящего подхода рекомендательной системы и алгоритма нечеткой
	кластеризации для решения проблемы;
	\item в результате полученных во время анализа данных разработать рекомендательную систему на основе нечеткой кластеризации;
	\item реализовать выбранный алгоритм нечеткой кластеризации;
	\item реализовать рекомендательную систему в программном продукте;
	\item провести исследование работоспособности реализованной рекомендательной системы.
\end{itemize}

\pagebreak