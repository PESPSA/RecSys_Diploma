\section*{ТЕРМИНЫ И УСЛОВНЫЕ ОБОЗНАЧЕНИЯ}

\textbf{Машинное обучение (Machine Learning, ML)} --- подраздел искусственного интеллекта, изучающий различные способы построения обучающихся алгоритмов. Среди множества парадигм и подходов в машинном обучении выделяются нейронные сети \cite{ClusterFO}. 

\textbf{Обучение без учителя (Unsupervised learning)} --- один из разделов машинного обучения. Изучает широкий класс задач обработки данных, в которых известны только описания множества объектов (обучающей выборки), и требуется обнаружить внутренние взаимосвязи, зависимости, закономерности, существующие между объектами.  

\textbf{Кластеризация (Сlustering)} --- задача группировки множества объектов на подмножества (кластеры) таким образом, чтобы объекты из одного кластера были более похожи друг на друга, чем на объекты из других кластеров по какому–либо критерию. 

\textbf{Нечеткая кластеризация (Fuzzy clustering)} --- форма кластеризации, в которой каждая точка данных может принадлежать более чем одному кластеру.

\textbf{Рекомендательные системы (Recommendation system)}  --- комплекс алгоритмов, программ и сервисов, задача которого предсказать, что может заинтересовать того или иного пользователя.  

\textbf{Холодный старт (Cold start)} --- ситуация, когда система не может делать никаких
выводов для пользователей или объектов, о которых она еще не собрала
достаточно информации.

\textbf{Датасет (Dataset)} --- набор данных, выборка.

\pagebreak